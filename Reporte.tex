\documentclass{article}
\usepackage{blindtext}
\usepackage{enumitem}
\usepackage[T1]{fontenc}
\usepackage[utf8]{inputenc}

\title{Compilador de C}
\author{Guerrero González Miguel Alberto}
\date{\today}

\begin{document}

\maketitle

\section*{Introducción}
De acuerdo a Wikipedia \cite{wikipedia_compilador}, un compilador es un tipo de traductor que transforma un programa entero de un lenguaje de programación (código fuente) a otro. Usualmente el lenguaje objetivo es código máquina.\\
La construcción de un compilador involucra la división del proceso en una serie de fases que variará con su complejidad. Generalmente estas fases se agrupan en dos tareas: el análisis del programa fuente y la síntesis del programa objeto.

\begin{description}
\item [Análisis] 



\item [Síntesis] 
\end{description}

\section*{Desarrollo}



\section*{Conclusiones}


\begin{thebibliography}{9}

\bibitem{wikipedia_compilador}
https://es.wikipedia.org/wiki/Compilador

\end{thebibliography}


\end{document}
