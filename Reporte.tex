\documentclass{article}
\usepackage{blindtext}
\usepackage{enumitem}
\usepackage{graphicx}
\usepackage{hyperref}
\hypersetup{
    colorlinks=true,
    linkcolor=blue,
    filecolor=magenta,      
    urlcolor=cyan,
}
\usepackage[T1]{fontenc}
\usepackage[utf8]{inputenc}

\title{Compilador de C}
\author{Guerrero González Miguel Alberto}
\date{\today}

\begin{document}

\maketitle

\section*{Introducción}

\subsection*{Compilador}
Un compilador es un tipo de traductor que transforma un programa entero de un lenguaje de programación (código fuente) a otro. Usualmente el lenguaje objetivo es código máquina.\cite{compilador_definicion}\\
La construcción de un compilador involucra la división del proceso en una serie de fases que variará con su complejidad. Generalmente estas fases se agrupan en dos tareas: el análisis del programa fuente y la síntesis del programa objeto.

\begin{description}

\item [Análisis] 
Se trata de la comprobación de la corrección del programa fuente, según la definición del lenguaje en términos de teoría de lenguajes formales. Incluye las siguientes fases:
\begin{itemize}
 \item Análisis léxico: descomposición del programa fuente en componentes léxicos.
 \item Análisis sintáctico: agrupación de los componentes léxicos en frases gramaticales.
 \item Análisis semántico: comprobación de la validez semántica de las sentencias aceptadas en la fase de análisis sintáctico.
\end{itemize}

\item [Síntesis] 
Su objetivo es la generación de la salida expresada en el lenguaje objeto y suele estar formado por una o varias combinaciones de fases de generación de código (normalmente código intermedio o código objeto) y de optimización de código (se busca obtener un programa objetivo lo más eficiente posible.

\end{description}

\subsection*{Flex}
Es una herramienta para generar escáneres: programas que reconocen patrones léxicos en un texto. Flex lee los ficheros de entrada dados, o la entrada estándar si no se le ha indicado ningún nombre de fichero, con la descripción de un escáner a generar. La descripción se encuentra en forma de parejas de expresiones regulares y código C, denominadas \textbf{reglas}. Flex genera como salida un fichero fuente en C, \emph{lex.yy.c}, que define una rutina \emph{yylex()}. Este fichero se compila y se enlaza con la librería \emph{-lfl} para producir un ejecutable. Cuando se arranca el fichero ejecutable, este analiza su entrada en busca de casos de las expresiones regulares. Siempre que encuentra uno, ejecuta el código C correspondiente. \cite{flex_definicion}

\subsection*{Bison}
Es un proyecto creado por GNU y es un programa que genera analizadores sintácticos con propósitos generales. \cite{bison_definicion}\\
Convierte la descripción de un lenguaje, escrita con una gramática libre LARL, en un programa C que sirve para la creación de un analizador sintáctico o parser.

\section*{Desarrollo}

\subsection*{Análisis léxico}
Para la parte del análisis léxico se tomaron las expresiones regulares de lenguaje C de la siguiente página web: https://www.lysator.liu.se/c/ANSI-C-grammar-l.html

\subsection*{Análisis sintáctico}
Para desarrollar el analizador sintáctico se tomó la gramática de lenguaje C de la siguiente página web: \href{https://www.lysator.liu.se/c/ANSI-C-grammar-y.html}



\section*{Conclusiones}


\begin{thebibliography}{9}

\bibitem{compilador_definicion}
https://es.wikipedia.org/wiki/Compilador

\bibitem{flex_definicion}
http://es.tldp.org/Manuales-LuCAS/FLEX/flex-es-2.5.html

\bibitem{bison_definicion}
https://es.wikipedia.org/wiki/BISON

\end{thebibliography}


\end{document}
